\documentclass{hogent-article}
\PaperTitle{De overgang van een poly-repository microservice structuur naar een mono-repository in een continue werkomgeving}
\PaperType{Paper Research Methods: onderzoeksvoorstel}
\Authors{Liam Goethals\textsuperscript{1}} % Authors
\CoPromotor{}
\affiliation{
  \textsuperscript{1} \href{mailto:liam.goethals@student.hogent.be}{liam.goethals@student.hogent.be}}
\Abstract{
Bij de implementatie van microservices wordt vaak teruggegrepen naar een poly-repository architectuur.
De alternatieve mono-repository architectuur is, afhankelijk van situatie, echter een even geschikte kandidaat.
Dit onderzoek bekijkt de bestaande implementatie mogelijkheden van mono-repositories en cre"eert een abstract stappenplan voor het migreren van poly-repositories naar mono-repositories. Dit stappenplan wordt onderbouwd aan de hand van praktijkkennis. Deze kennis wordt vergaard door middel van de uitwerking van een Proof-of-Concept. Uiteindelijk wordt er verwacht dat de migratie uitvoerbaar is zonder de nood aan een team van experten en dat een abstract stappenplan kan opgesteld worden aan de hand van de bekomen kennis.
}
\Keywords{Applicatieontwikkeling (andere); Mono-repository; Strategie; Versiebeheer}
\newcommand{\keywordname}{Sleutelwoorden}
\begin{document}
\flushbottom
\maketitle
\tableofcontents
\thispagestyle{empty}
\section{Inleiding}
In het landschap van microservices wordt vaak gebruik gemaakt van poly-repository architecturen. Vele bronnen en stappenplannen op het internet bespreken hoe dit soort architectuur opgezet en onderhouden moet worden. Het opsplitsen van services en het onderscheiden van applicaties in verschillende repositories zijn echter twee aparte zaken.

Dit onderzoek beschrijft een bestaand alternatief voor de populaire implementatie, namelijk: mono-repositories. Daarnaast wordt onderzocht hoe de overschakeling van poly-repositories naar mono-repositories kan plaatsvinden, zonder de continue voortgang van het bestaande project te verbreken. 
Dit onderzoek zal zich niet richten op de waarom-vraag van het overschakelen aangezien deze beslissing sterk afhankelijk is van context. 

Aan de hand van een bedrijfscasus wordt de migratie van een bestaande poly-repository architectuur naar een mono-repository architectuur uitgevoerd. Tot slot wordt een abstract stappenplan opgesteld dat ontwikkelaars moet helpen bij de architecturale overschakeling in hun specifieke bedrijfscontext.

\section{Overzicht literatuur}
Het doel van microservices is het opsplitsen van een complexe applicatie in een aantal sub-applicaties met elk een specifieke verantwoordelijkheid \autocite{Thoenes2015}. Anders dan bij een monoliete applicatie, is elke service een alleenstaande entiteit die op zichzelf kan bestaan. Deze aparte entiteiten kunnen elk een eigen team van developers hebben waardoor een gelijktijdige samenwerking van meerdere teams aan hetzelfde hoofddoel mogelijk is. Elk team is dan verantwoordelijk voor de correcte werking, testing en deployment van hun service.
Hoewel microservices niet enkel beperkt zijn tot web toepassingen, is de architectuur grotendeels terug te vinden in de wereld van webapplicaties \autocite{Richardson2019}.

De concrete implementatie van microservices is sterk afhankelijk van de bedrijfscontext. In de praktijk worden de verschillende services vaak opgesplitst in afzonderlijke repositories (polyrepos) \autocite{Brousse2019}. Elke repository bevat vervolgens normaliter applicatiespecifieke configuraties zoals: testen, een eigen mappenstructuur, CI/CD pipelines, toegangsrechten en afhankelijk van de benodigdheden en gebruikte systemen nog veel meer andere zaken. Deze implementatie vorm heeft als gevolgd dat er duplicate code en inconsistenties kunnen voorkomen doorheen de verschillende applicaties. Wanneer twee of meerdere applicaties een onderlinge afhankelijk hebben, zullen aanpassingen in de \'e\'ene applicatie mogelijks onverwachte effecten hebben in de andere(n). Dit laatste fenomeen maakt versiebeheer een belangrijk concept binnen opgesplitste services \autocite{Richardson2019}.

Een aantal grote bedrijven waaronder Google \autocite{Potvin&Levenberg2016}, Microsoft \autocite{Tirion2021} en Facebook \autocite{Goode2014} besloten echter voor een andere implementatie vorm dan polyrepos, namelijk: de mono-repository (monorepo) architectuur.
In een monorepo worden services alsnog opgesplitst, maar in plaats van te onderverdelen in verschillende repositories worden ze onderverdeeld in submappen binnen eenzelfde repository.
Alle afzonderlijke services kunnen nog steeds direct aangesproken worden en kunnen dus beschouwd worden als alleenstaande applicaties.
Dit laatste onderscheidt monorepos van monoliete applicaties \autocite{Brito2018}.

Voor de gepaste implementatie keuze bestaan er voor beide vormen onderbouwde argumenten. Zolang de microservice methodologi"en gerespecteerd worden kunnen echter met beide een correct resultaat behaald worden. Elke implementatie vorm heeft zijn eigen voor- en nadelen op vlak van ontwikkelingsproces maar de keuze blijft afhangen van preferentie en context. Volgens een artikel van (ex-)medewerkers \textcite{Potvin&Levenberg2016} heeft Google zijn keuze grotendeels gebaseerd op de versimpelde werking van onderlinge service afhankelijkheid. In het verleden is er echter al een aantal keer intern onderzocht of het gunstig kon zijn om af te stappen van monorepos.  

Zowel bij polyrepos als bij monorepos draagt elke service zijn eigen verantwoordelijkheid. Een publiek aansprekingspunt geeft de andere services toegang tot de blootgestelde functionaliteiten van een bepaalde service. Voor beide architecturen verloopt de communicatie extern, via bijvoorbeeld REST API's. Wat anders is dan de communicatie tussen componenten in een monoliete applicatie waar de communicatie intern zal verlopen \autocite{Richardson2019}.

Binnen microservices bestaan er verschillende deployment strategie"en. Opnieuw is de keuze afhankelijk van bedrijfscontext. Ongeacht de gekozen strategie is het binnen de microservice filosofie belangrijk dat code snel en getest op de productieomgeving kan worden gezet \autocite{Richardson2019}.
Bij polyrepos zal een wijziging op service niveau betekenen dat de repository zelf opnieuw gedeployed moet worden. Wanneer dezelfde strategie echter toegepast zou worden bij monorepos kan dit voor ongewenste problemen zorgen. Indien dit voor de gehele repository zou gebeuren bijvoorbeeld dan zal dit tot gevolg hebben dat alle services opnieuw gedeployed worden. Als in dit geval het totale project een zeer grote omvang heeft, zoals dit bij Google het geval is \autocite{Potvin&Levenberg2016}, zullen er veel overbodige deployment processen van start gaan. Om dit te voorkomen zullen monorepo-specifieke tools de interne services afzonderlijk deployen \autocite{Matei2020}. 

Zoals eerder vermeld is ook het snel kunnen testen van de code een belangrijk principe binnen de wereld van microservices. Analoog aan het overbodig deployen van ongewijzigde code, zullen ook de testen dynamisch gestart moeten kunnen worden om een gelijke werking aan polyrepos te kunnen garanderen. 

Voor polyrepos volstaat het vaak om gebruik te maken van een klassiek versiebeheersysteem. Ook voor monorepos zouden deze systemen in theorie bruikbaar moeten zijn.  
Uit het eerder vermelde artikel van (ex-)medewerkers \textcite{Potvin&Levenberg2016} blijkt echter dat Google een eigen versiebeheersysteem gebruikt, genaamd Piper, voor het bijhouden van de individuele bestandswijzigingen in hun monorepo. Google koos, volgens hen, voor het ontwikkelen van een eigen tool omdat de bestaande systemen tekortkwamen voor grote monorepos.  Ook voor Facebook bleek dit een probleem te vormen wanneer zij gebruiken wouden maken van het populaire versiebeheersysteem Git\footnote{https://git-scm.com/}. In plaats van een eigen tool te ontwikkelen besloot Facebook om een bestaand systeem genaamd Mercurial\footnote{https://www.mercurial-scm.org/} te verbeteren voor hun toepassing \autocite{Goode2014}.
Het valt op te merken dat de repositories van Google en Facebook beide een uitzonderlijk grote omvang hebben.

Op het internet zijn vele monorepo-specifieke beheer- en deploy tools te vinden. Zo is Pants\footnote{https://v1.pantsbuild.org/index.html} bijvoorbeeld een open-source, door Twitter gecre"eerde, monorepo tool dat ontwikkelaars toelaat om plugins te implementeren voor om het even welke taal \autocite{Olson2016}. Hoewel er tools bestaan die taal-agnostisch te werk kunnen gaan, zoals Pants, bestaan er ook een groot deel taal-specifieke tools. Lerna\footnote{https://github.com/lerna/lerna} is een voorbeeld van een populaire tool die enkel ontwikkeld is voor Javascript en Typescript. De keuze van de tool, als deze nodig blijkt binnen de context, zal afhankelijk zijn van de specifieke voorkeuren van een bedrijf of organisatie.

Ondanks dat de meeste tools een starter gids voorzien op hun website, is deze steeds gericht op het cre"eeren van een nieuw project. De monorepo tool Nx\footnote{https://nx.dev/migration/manual} heeft als uitzondering wel een kort platform specifiek stappenplan ter beschikking dat terug te vinden is in hun documentatie. Daarnaast bestaan er een aantal specifieke blogposts die zich vooral focussen op het cre"eren van migratie scripten. Migreren van een bestaande architectuur naar een monorepo is een complex proces waarvoor een abstract stappenplan ontbreekt. Dit laat de overschakeling echter onmogelijker uitstralen dan ze effectief zou kunnen zijn.

\section{Methodologie}
Het onderzoek bestaat uit zes fases. 
De totale duur van het onderzoek bedraagt 13 weken.

Fase \'e\'en is het uitvoeren van een literatuurstudie. Er wordt opzoekingswerk verricht naar de bestaande mono-repository technologi"en en literatuur omtrent mono-repository migraties. Voor deze fase is de geschatte duurtijd: twee weken. 

Fase twee is de aanvang van de Proof-of-Concept in samenwerking met het bedrijf. In deze fase wordt de interne structuur van het bedrijf bestudeerd en geschetst om een beeld te kunnen scheppen van de huidige situatie en de nodige vereisten. Er wordt extra aandacht besteed aan het bestuderen van onderlinge afhankelijkheid en het huidige gebruik van databanksystemen. Deze schets brengt enkel de services in kaart die relevant zijn voor het onderzoek. Voor deze fase is de geschatte duurtijd: twee weken.

Fase drie bestaat uit de theoretische implementatie van de verworven kennis uit de literatuurstudie en het plannen van de implementatie. Hier wordt in samenspraak met het bedrijf gekeken naar welke implementatie vorm de voorkeuren uitgaan. De reeds verkozen taal, architectuur, databanksystemen, toekomstplannen, systeemvereisten en schaalbaarheids eisen zullen impact hebben op de te maken beslissing. Voor deze fase is de geschatte duurtijd: drie weken.

Fase vier is de effectieve implementatie van het vooropgestelde plan. Als Proof-of-Concept volstaat het om een aantal repositories te migreren naar een mono-repository. Een volledige migratie zou teveel werk en teveel toegang tot interne bedrijfsgegevens vereisen. Deze fase vergt een zekere voorbereidingstijd waarbij het opzetten van backup systemen essentieel is. Stap voor stap zullen de gekozen repositories ge"implementeerd worden in de mono-repository. De migratie van deze repository gebeurt 'e'en voor 'e'en waarbij pas overgeschakeld kan worden op de volgende repository wanneer de huidige volledig ge"implementeerd is en functioneert zoals voordien. Voor deze fase is de geschatte duurtijd: vier weken.

Fase vijf vindt plaats na de implementatie. In deze fase wordt gereflecteerd op het bekomen resultaat en worden de bevindingen gebundeld. Er wordt een oplijsting gemaakt van onverwachte problemen en er wordt bekeken of deze al dan niet te voorkomen waren. Voor deze fase is de geschatte duurtijd: \'e\'en week.

De zesde en laatste fase bestaat uit het opstellen van een abstract stappenplan aan de hand van de verworven kennis doorheen het onderzoek. Hiervoor wordt de praktisch ervaring gecombineerd met de vooraf bestudeerde literatuur. Voor deze fase is de geschatte duurtijd: \'e\'en week.

\section{Verwachte conclusies}

Er wordt verwacht dat het mogelijk moet zijn om een abstract stappenplan op te stellen. Indien blijkt dat de verworven kennis doorheen het onderzoek ondergeschikt blijkt voor het opstellen van een volledig stappenplan, dan wordt verwacht dat deze tenminste dienst kan doen als een katalysator voor verder onderzoek. 
Ook wordt er verwacht dat de praktische implementatie in de meeste bedrijfscontexten uitvoerbaar is en dat het Proof-of-Concept gedeelte van het onderzoek dus tot een positief einde gebracht zal worden. Indien blijkt dat voor de gebruikte technologie binnen het bedrijf geen direct equivalent bestaat binnen de wereld van mono-repositories kan de ingeschatte tijd mogelijks te optimistisch blijken.

\phantomsection
\footnotetext{https://github.com/liamgoethals/paper-research-methods}
\printbibliography[heading=bibintoc]

\end{document}
